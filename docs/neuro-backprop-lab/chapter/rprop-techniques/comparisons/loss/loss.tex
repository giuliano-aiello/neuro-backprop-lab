\begin{subsection}{Loss}
    \begin{figure}[h!]
        \centering
        \begin{subfigure}[b]{0.48\textwidth}
            \centering
            \includesvg[width=\textwidth]{training_loss.svg}
            \label{fig:training_loss}
        \end{subfigure}
        \hfill
        \begin{subfigure}[b]{0.48\textwidth}
            \centering
            \includesvg[width=\textwidth]{evaluation_loss.svg}
            \label{fig:evaluation_loss}
        \end{subfigure}
    \end{figure}
    \par It can be instantly observed that the different versions of~\glsxtrshort{rprop} exhibit the same loss trend in the training phase: during the initial epochs, a steep decrease in loss is followed by a gradual decrease that starts from about the tenth epoch and persists until the last epoch.\\

    Across these experiments, there is not a single technique that clearly stands out, they almost seem to overlap, although it can be seen that~\glsxtrshort{rprop}\textsuperscript{+} slightly exhibits a worse loss.\\
    \par Almost the same applies during the evaluation phase. However, a slight but greater discrepancy can be observed between each of the~\glsxtrshort{rprop} versions from the point at which the gradual decrease begins.\\ The~\glsxtrshort{rprop}\textsuperscript{-} implementation seems to achieve the lowest error among all. Whereas, again, the~\glsxtrshort{rprop}\textsuperscript{+} version exhibits the greatest error from the start of the gradual decrease onwards.
    \par The evidence suggests that an early stopping criterion would not have been beneficial, as the evaluation loss does not exhibit any increasing curve from a certain point onwards. Even though it seems that the evaluation loss would have stopped improving from the $50$\textsuperscript{th} epoch onwards.\\

    \par An important remark is made on the difference in loss between training and evaluation phases: they differ by approximately an order of magnitude, this could be a warning of overfitting that could be worth monitoring.
\end{subsection}
\clearpage