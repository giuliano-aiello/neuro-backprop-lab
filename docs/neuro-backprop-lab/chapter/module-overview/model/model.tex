\begin{section}{Model}
    \par This class, \texttt{model.py}, represents the artificial neural network model architecture to be trained and tested. What follows are empirical choices that are the result of various experiments.\\
    \par The model is a shallow~\glsxtrlong{mlp} based on \texttt{torch.nn.Module}\footnote{\textit{https://pytorch.org/docs/stable/generated/torch.nn.Module.html} (accessed 2025)} class. Its three layers are fully connected using \texttt{torch.nn.Linear(.)} and they feed forward as follows:
    \begin{enumerate}
        \item the first layer flattens the input~\glsxtrshort{mnist} image, by transforming it from a multidimensional vector to a $784$-sized (since a $28 \times 28$-sized image is manipulated) one-dimensional vector;
        \item the hidden layer receives the transformed vector and processes it into a $128$-sized vector with a \glsxtrshort{relu} activation function to introduce non-linearity;
        \item the output layer extracts the final predictions by transforming the $128$-sized vector into a $10$-sized vector, which corresponds to the number of possible classes for classification.
    \end{enumerate}
    \vspace{1cm}
    \begin{figure}[h]
        \centering
        \includesvg[inkscapelatex=false,width=200pt]{model.svg}
        \caption[A schematic representation of the model.]{\emph{A schematic representation of the model}\footnotemark}
        \label{fig:model}
    \end{figure}

    \footnotetext{Representation generated using NN-SVG, a public tool available on GitHub (\textit{https://github.com/alexlenail/NN-SVG})}
\end{section}
\clearpage