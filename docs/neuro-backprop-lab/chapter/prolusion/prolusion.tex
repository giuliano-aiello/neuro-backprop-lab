\begin{section}{Model}
    \par This class, \texttt{model.py}, represents the artificial neural network model architecture to be trained and tested. What follows are empirical choices that are the result of various experiments.\\
    \par The model is a shallow~\glsxtrlong{mlp} based on \texttt{torch.nn.Module}\footnote{\textit{https://pytorch.org/docs/stable/generated/torch.nn.Module.html} (accessed 2025)} class. Its three layers are fully connected using \texttt{torch.nn.Linear(.)} and they feed forward as follows:
    \begin{enumerate}
        \item the first layer flattens the input~\glsxtrshort{mnist} image, by transforming it from a multidimensional vector to a $784$-sized (since a $28 \times 28$-sized image is manipulated) one-dimensional vector;
        \item the hidden layer receives the transformed vector and processes it into a $128$-sized vector with a \glsxtrshort{relu} activation function to introduce non-linearity;
        \item the output layer extracts the final predictions by transforming the $128$-sized vector into a $10$-sized vector, which corresponds to the number of possible classes for classification.
    \end{enumerate}
    \vspace{1cm}
    \begin{figure}[h]
        \centering
        \includesvg[inkscapelatex=false,width=200pt]{model.svg}
        \caption[A schematic representation of the model.]{\emph{A schematic representation of the model}\footnotemark}
        \label{fig:model}
    \end{figure}

    \footnotetext{Representation generated using NN-SVG, a public tool available on GitHub (\textit{https://github.com/alexlenail/NN-SVG})}
\end{section}
\clearpage
\clearpage
\begin{chapter}{Prolusion}
    \begin{section}{Goal}
        \par This report provides a comprehensive overview of a Python project whose goal is to develop and compare different parameter optimization algorithms involved in a machine learning process, as~\glsxtrfull{rprop}. \glsxtrshort{mnist} is the target of the learning model.
		\par The project follows the ``Empirical evaluation of the improved Rprop learning algorithms'' article by Christian Igel and Michel Hüsken (2001).
    \end{section}
    \newpage
	\begin{section}{Software Stack}
		\begin{itemize}
			\item Python 3.9.6
			\item PyTorch 2.6.0
		\end{itemize}
		The project is equipped with a \texttt{requirements.txt} file which allows for seamless installation of dependencies, by executing \texttt{pip install -r requirements.txt}.
	\end{section}
	\newpage
	\begin{section}{Project Structure}
		\dirtree{%
			.1 neuro-backprop-lab/.
			.2 model/.
			.2 tester/.
			.3 tester.py.
			.3 {<}trained\_model{>}.pt.
			.2 trainer/.
			.3 irpropplus/.
			.3 rpropminus/.
			.3 rpropplus/.
			.3 trainer.py.
			.2 utils/.
			.3 loader\_dataset.py.
			.2 test\_model.py.
			.2 train\_model.py.
		}
		\medskip
		\begin{itemize}
			\item \texttt{model} includes the neural network model architecture.
			\item \texttt{tester} handles the testing flow of the ready-to-use \texttt{<trained\_model>.pt}.
			\item \texttt{trainer} handles the examined backpropagation techniques and the training flow of the model, saving it as \texttt{<trained\_model>.pt}.
			\item \texttt{utils} offers utility functions designed to support the root project scripts.
		\end{itemize}
	\end{section}
\end{chapter}